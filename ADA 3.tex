\documentclass[12pt]{article}
\usepackage{amsmath,amssymb}
\usepackage{amsmath, array}
\usepackage{amsmath,amsthm, amssymb, latexsym}
\usepackage{mathtools}
\usepackage[breaklinks=true]{hyperref}
\hypersetup{% 
	pdfborder = {0 0 0} 
} 
\usepackage{graphicx}
\usepackage{float}
\usepackage[spanish]{babel}
\usepackage[utf8]{inputenc}
\usepackage{multirow}
\usepackage{enumerate}
\usepackage{setspace}
\usepackage{subfigure}


\begin{document}
	\begin{titlepage}
		
		\begin{center}
			\vspace*{-1in}
			\begin{figure}[htb]
				\begin{center}
					\includegraphics[width=8cm]{logo}
				\end{center}
			\end{figure}
			
			Facultad de Ingeniería\\
			\vspace*{0.15in}
			Licenciatura en Ingeniería Física \\
			\vspace*{0.2in}
			\begin{large}
				Alan Mosqueda Camacho\\
				Carmen Andrea Rivera Martínez\\
				Gonzalo Herrera Ramirez\\
				Jesús Alejandro Salazar González\\
				José Israel Cetina Palomo\\
				Pedro Felipe Baeza Ortiz\\
			\end{large}
			\vspace*{0.2in}
			\begin{Large}
				\textbf{ADA 3: Ejercicios} \\
			\end{Large}
			\vspace*{0.2in}
			\begin{large}
				Fisicoquímica \\
			\end{large}
			\vspace*{0.3in}
			\rule{80mm}{0.1mm}\\
			\vspace*{0.1in}
			\begin{large}
				Maestro: Avel Adolfo González Sánchez\\
			\end{large}
		\end{center}
		
	\end{titlepage}
\section*{Ejercicio 1}

\section*{Ejercicio 2}
Las entalpias de combustión de la glucosa $(\mathrm{C}_6\mathrm{H}_{12}\mathrm{O}_6)$ y etanol $(\mathrm{C}_2\mathrm{H}_5\mathrm{OH})$ son $-2815\frac{\mathrm{kJ}}{\mathrm{mol}}$ y $1372 \frac{\mathrm{kJ}}{\mathrm{mol}}$, respectivamente. Con estos datos determina la energía intercambiada en la fermentación de un mol de glucosa, reacción en la que se produce etanol y $\mathrm{CO}_2$ ¿es exotérmica la reacción?\\
\\
Reacción de combustión de la glucosa

\begin{displaymath}
	\mathrm{C}_6\mathrm{H}_{12}\mathrm{O}_6+6\mathrm{O}_2 \rightarrow 6\mathrm{O}_2+6\mathrm{H}_2\mathrm{O} \;\;\;\;\;\;\; \Delta\mathrm{H}_1^{\circ}=-2815\frac{\mathrm{kJ}}{\mathrm{mol}}
\end{displaymath}

Reacción de combustión del etanol

\begin{displaymath}
	\mathrm{O}_2\mathrm{H}_5\mathrm{OH}+3\mathrm{O}_2 \rightarrow 2\mathrm{CO}_2+3\mathrm{H}_2\mathrm{O}\;\;\;\;\;\;\; \Delta\mathrm{H}_2^{\circ}=-1372\frac{\mathrm{kJ}}{\mathrm{mol}}
\end{displaymath}

Reacción de fermentación

\begin{displaymath}
		\mathrm{C}_6\mathrm{H}_{12}\mathrm{O}_6 \rightarrow 2\mathrm{C}_2\mathrm{H}_5\mathrm{OH} + 2\mathrm{CO}_2 \;\;\;\;\;\;\; \Delta\mathrm{H}_3^{\circ}= ?
\end{displaymath}

Por lo tanto para $\Delta\mathrm{H}_3^{\circ}$ se tiene

\begin{displaymath}
	\Delta\mathrm{H}_3^{\circ}=\Delta\mathrm{H}_1^{\circ}-2\Delta\mathrm{H}_2^{\circ}
\end{displaymath}

donde el 2 que acompaña a $\Delta\mathrm{H}_2^{\circ}$ se debe a que en la reacción de fermentación hay dos moles de etanol, lo que nos da como resultado:

\begin{displaymath}
	\Delta\mathrm{H}_3^{\circ}=-71\frac{\mathrm{kJ}}{\mathrm{mol}}
\end{displaymath}

Como el signo es negativo, hay pérdida de energía, es decir, es una reacción exotérmica
\newpage
\section*{Ejercicio 3}

Calcular el calor de formación del ácido metanoico $(\mathrm{H}-\mathrm{COOH})$, a partir de los siguientes calores de reacción.\\
\\
Datos:

\begin{displaymath}
	\mathrm{C} + \frac{1}{2}\mathrm{O}_2\rightarrow\mathrm{CO} \;\;\;\;\;\;\; \Delta\mathrm{H}_f^{\circ}=-110.4\frac{\mathrm{kJ}}{\mathrm{mol}}
\end{displaymath}

\begin{displaymath}
	\mathrm{H}_2+\frac{1}{2}\mathrm{O}_2 \rightarrow \mathrm{H}_2\mathrm{O} \;\;\;\;\;\;\; \Delta\mathrm{H}_f^{\circ}=-285.5\frac{\mathrm{kJ}}{\mathrm{mol}}
\end{displaymath}

\begin{displaymath}
	\mathrm{CO} + \frac{1}{2}\mathrm{O}_2 \rightarrow \mathrm{CO}_2 \;\;\;\;\;\;\; 
	\Delta\mathrm{H}_f^{\circ}=-283\frac{\mathrm{kJ}}{\mathrm{mol}}
\end{displaymath}

\begin{displaymath}
	\mathrm{H}-\mathrm{COOH}+\frac{1}{2}\mathrm{O}_2 \rightarrow \mathrm{CO}_2 + \mathrm{H}_2\mathrm{O} \;\;\;\;\;\;\; \Delta\mathrm{H}_{comb}^{\circ}=-259.6\frac{\mathrm{kJ}}{\mathrm{mol}}
\end{displaymath}

Se necesita determinar el calor de formación del $\mathrm{CO}_2$, por lo tanto se parte desde la ecuación siguiente:

\begin{displaymath}
		\mathrm{CO} + \frac{1}{2}\mathrm{O}_2 \rightarrow \mathrm{CO}_2
\end{displaymath}

Por lo tanto

\begin{displaymath}
	\Delta\mathrm{H}_{comb}^{\circ}=\sum n_p\Delta\mathrm{H}_f^{\circ}-\sum n_r\Delta\mathrm{H}_f^{\circ}
\end{displaymath}

\begin{displaymath}
	\Rightarrow \Delta\mathrm{H}_{comb}^{\circ}= \Delta\mathrm{H}_f^{\circ}(\mathrm{CO}_2)-\Delta\mathrm{H}_f^{\circ}(\mathrm{CO})
\end{displaymath}

\begin{displaymath}
	\Rightarrow -\Delta\mathrm{H}_f^{\circ}(\mathrm{CO}_2)=-\Delta\mathrm{H}_f^{\circ}(\mathrm{CO})-\Delta\mathrm{H}_{comb}^{\circ}
\end{displaymath}

\begin{displaymath}
	\Rightarrow -\Delta \mathrm{H}_f^{\circ}(\mathrm{CO}_2)=\left[ -(-110.4) - (-283) \right] \frac{\mathrm{kJ}}{\mathrm{mol}}
\end{displaymath}

\begin{displaymath}
	\therefore \Delta\mathrm{H}_f^{\circ}(\mathrm{CO}_2)=-393.4 \frac{\mathrm{kJ}}{\mathrm{mol}}
\end{displaymath}
\newpage
Ahora se calcula el calor de formación del ácido metanoico partiendo de la siguiente fórmula:

\begin{displaymath}
	\mathrm{H}-\mathrm{COOH}+\frac{1}{2}\mathrm{O}_2 \rightarrow \mathrm{CO}_2 + \mathrm{H}_2\mathrm{O}
\end{displaymath}

De manera similar a lo anteriormente mostrado obtenemos la siguiente fórmula:

\begin{displaymath}
	\Delta\mathrm{H}_f^{\circ}(\mathrm{H}-\mathrm{COOH})=\Delta\mathrm{H}_f^{\circ}(\mathrm{CO}_2)+\Delta\mathrm{H}_f^{\circ}(\mathrm{H}_2\mathrm{O})-\Delta\mathrm{H}_{comp}^{\circ}
\end{displaymath}

Sustituyendo y resolviendo obtenemos:

\begin{displaymath}
	\therefore \Delta\mathrm{H}_f^{\circ}(\mathrm{H}-\mathrm{COOH})=-419.13\frac{\mathrm{kJ}}{\mathrm{mol}}
\end{displaymath}
\section*{Ejercicio 4}

\section*{Ejercicio 5}

Si cuando se forma 1 gramo de metanol $(\mathrm{CH}_3\mathrm{OH})$ se desprenden $7.46$ kilojulios, calcula:

\begin{enumerate}[a)]
	\item ¿Cuál será el valor de su entalía de formación?
	
	\item ¿Cuál será la entalpía estandar de combustión del metanol utilizando la ley de Hess?
\end{enumerate}

Datos: Masa atómica

\begin{itemize}
	\item $\mathrm{C}=12u$
	
	\item $\mathrm{O}=16u$
	
	\item $\mathrm{H}=1u$
\end{itemize}

Entalpías estandar de formación del $\mathrm{CO}_2(g)$ y del $\mathrm{H}_2\mathrm{O}(l)$, respectivamente: $-393\frac{\mathrm{kJ}}{\mathrm{mol}}$ y $-285.8\frac{\mathrm{kJ}}{\mathrm{mol}}$\\
\\
\textbf{Solución a):}\\
\\
Debido a que por cada gramo de metanol se liberan 7.46$\mathrm{kJ}$, entonces la entalpía de formación será:

\begin{displaymath}
	\Delta\mathrm{H}^{\circ}(\mathrm{CH}_3\mathrm{OH})=-7.46\frac{\mathrm{kJ}}{\mathrm{g}}
\end{displaymath}

Para convertir la expresión en $\frac{\mathrm{kJ}}{\mathrm{mol}}$ se necesita encontrar la masa molar del metanol, esta se define como la suma de las masas atómicas de los elementos que la componen, por lo tanto:

\begin{displaymath}
	M_{\mathrm{CH}_3\mathrm{OH}}=M_{\mathrm{C}}+3M_{\mathrm{H}}+M_{\mathrm{O}}+M_{\mathrm{H}}=12u+4(1u)+16u
\end{displaymath}

\begin{displaymath}
	M_{\mathrm{CH}_3\mathrm{OH}}=32u=32\frac{\mathrm{g}}{\mathrm{mol}}
\end{displaymath}

Por cada 32 gramos de metanol existe 1 mol de metanol, por lo tanto la entalpía de formación será:

\begin{displaymath}
	\Delta\mathrm{H}^{\circ}(\mathrm{CH}_3\mathrm{OH})=-7.46\frac{\mathrm{kJ}}{\mathrm{g}} \left( 32 \frac{\mathrm{g}}{\mathrm{mol}} \right) = -238.72 \frac{\mathrm{kJ}}{\mathrm{mol}}
\end{displaymath}

\textbf{Solución b)}\\
\\
La ley de Hess establece:

\begin{displaymath}
	\Delta\mathrm{H}_{comp}^{\circ}=\sum n_p\Delta\mathrm{H}_i^{\circ}-\sum n_r	Delta\mathrm{H}_i^{\circ}
\end{displaymath}

Por tanto

\begin{displaymath}
	\mathrm{CH}_3\mathrm{OH}+\frac{3}{2}\mathrm{O}_2(g) \rightarrow \mathrm{CO}_2(g)+2\mathrm{H}_2\mathrm{O}(l)
\end{displaymath}

Sustituyendo:

\begin{displaymath}
	\Delta\mathrm{H}_{comb}^{\circ}=\Delta\mathrm{H}^{\circ}(\mathrm{CO}_2)+2\Delta\mathrm{H}^{\circ}(\mathrm{H}_2\mathrm{O})-\Delta\mathrm{H}^{\circ}(\mathrm{CH}_3\mathrm{OH})
\end{displaymath}

\begin{displaymath}
	\Delta\mathrm{H}_{comb}^{\circ}=[-393.8+2(285.8)-(-238.72)]\frac{\mathrm{kJ}}{\mathrm{mol}}
\end{displaymath}

\begin{displaymath}
	\therefore \Delta\mathrm{H}_{comb}^{\circ}=-726.68\frac{\mathrm{kJ}}{\mathrm{mol}}
\end{displaymath}












\end{document}